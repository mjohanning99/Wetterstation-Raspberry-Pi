\documentclass[a4paper,12pt]{article}

\usepackage{fancyhdr}
\usepackage[german]{babel}
\usepackage{microtype}
\usepackage{graphicx}
\usepackage{wrapfig}
\usepackage{enumitem}
\usepackage{amsmath}
\makeindex

\begin{document}

\pagenumbering{gobble}
\title{\textbf{Raspberry Pi Wetterstation}}
\author{Von Leon Lepa, Oliver Winkler und Marvin Johanning}
\date{April 2020}
\maketitle
\abstract{Dieses Dokument dient zur Dokumentation des Aufbaus sowie der Programmierung einer Wetterstation, die mithilfe eines Raspberry Pis verwirklicht wurde. Zudem wird die Durchführung sowie die verwendeten Hard- und Softwarekomponenten beschrieben.}
\pagebreak

\tableofcontents
\pagebreak

\setlength{\headheight}{15.2pt}
\pagestyle{fancy}
\renewcommand{\headrulewidth}{2pt}
\renewcommand{\footrulewidth}{2pt}
\fancyhead[L]{Raspberry Pi Wetterstation}

\pagenumbering{roman}
\section{Einleitung und Übersicht}
Als Projekt wurde eine Wetterstation deshalb ausgewählt, da private Wetterstationen die Wetterverhältnisse an einem bestimmten Standort besser ermitteln können, als beispielsweise diejenigen Wetterstationen, die von verschiendensten Online-Diensten wie z. B. "`Google"' oder "`wetter.de"' genutzt werden.

Die Daten dieser Wetterstationen stammen oftmals aus einem anderen Ort, der gegebenenfalls sogar mehrere Kilometer weit entfernt liegt, wodurch die Genauigkeit dieser Daten oftmals schwankt. Ein bekanntes Problem ist beispielsweise die inkorrekte Anzeige von Regen an einem Standort, obwohl dort die Sonne scheint.

Der Besitz einer privaten Wetterstation jedoch macht es möglich, die genauen Wetterverhältnisse an einem exakt festgelegten Standort zu ermitteln. Die dadurch entstehenden Möglichkeiten sind vielfältig, wie beispielsweise die Errechnung der Median- oder Durchschnittstemperatur oder -luftfeuchtigkeit.

Ein weiterer Vorteil besteht darin, die Temperatur- und Luftfeuchtigkeitsdaten in einem geschlossenen Raum ermitteln zu können und bei Überschreiten eines Grenzwertes z. B. einen Alarm erschallen zu lassen. Dies könnte beispielsweise in Gewächshäusern oder auch Serverräumen zum Einsatz kommen und somit möglicherweise teuren Hardwaretausch — aufgrund von Überhitzung — entgegenkommen.

Die "`Raspberry Pi Wetterstation"' bietet hierfür eine geeignete Grundlage; sie ermittelt die Wetterdaten anhand von Sensoren — dessen genaue Bezeichnungen und Funktionen im späteren Verlaufe dieser Dokumentation noch ausführlicher erklärt werden — und gibt diese auf einem LCD-Bildschirm aus. Weitere Funktionen lassen sich aufgrund des einfach zu programmierenden Raspberry Pi ohne großen Aufwand hinzufügen.

Im folgenden werden nun die für die Verwirklichung des Projektes verwendeten Hard- und Softwarekomponenten genauer beschrieben; daraufhin wird die Durchführung und Planung des Projektes erläutert.
\pagebreak

\subsection{Verwendete Programme}
\pagebreak

\subsection{Verwendete Hardware}

\pagebreak
\section{Projektdurchführung}

\end{document}
